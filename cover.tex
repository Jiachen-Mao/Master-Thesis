% -*-latex-*-
% 
% For questions, comments, concerns or complaints:
% thesis@mit.edu
% 
%
% $Log: cover.tex,v $
% Revision 1.8  2008/05/13 15:02:15  jdreed
% Degree month is June, not May.  Added note about prevdegrees.
% Arthur Smith's title updated
%
% Revision 1.7  2001/02/08 18:53:16  boojum
% changed some \newpages to \cleardoublepages
%
% Revision 1.6  1999/10/21 14:49:31  boojum
% changed comment referring to documentstyle
%
% Revision 1.5  1999/10/21 14:39:04  boojum
% *** empty log message ***
%
% Revision 1.4  1997/04/18  17:54:10  othomas
% added page numbers on abstract and cover, and made 1 abstract
% page the default rather than 2.  (anne hunter tells me this
% is the new institute standard.)
%
% Revision 1.4  1997/04/18  17:54:10  othomas
% added page numbers on abstract and cover, and made 1 abstract
% page the default rather than 2.  (anne hunter tells me this
% is the new institute standard.)
%
% Revision 1.3  93/05/17  17:06:29  starflt
% Added acknowledgements section (suggested by tompalka)
% 
% Revision 1.2  92/04/22  13:13:13  epeisach
% Fixes for 1991 course 6 requirements
% Phrase "and to grant others the right to do so" has been added to 
% permission clause
% Second copy of abstract is not counted as separate pages so numbering works
% out
% 
% Revision 1.1  92/04/22  13:08:20  epeisach

% NOTE:
% These templates make an effort to conform to the MIT Thesis specifications,
% however the specifications can change.  We recommend that you verify the
% layout of your title page with your thesis advisor and/or the MIT 
% Libraries before printing your final copy.
\title{Automatic calibration of an urban microclimate model under uncertainty \vspace{12pt}}

\author{Jiachen Mao}
\prevdegrees{B.Eng., Tongji University (2015) \vspace{12pt}}
\department{Department of Architecture}

\degree{\vspace{12pt} Master of Science in Building Technology \vspace{12pt}}

% If you wish to list your previous degrees on the cover page, use the 
% previous degrees command:
%       \prevdegrees{A.A., Harvard University (1985)}
% You can use the \\ command to list multiple previous degrees
%       \prevdegrees{B.S., University of California (1978) \\
%                    S.M., Massachusetts Institute of Technology (1981)}

% If the thesis is for two degrees simultaneously, list them both
% separated by \and like this:
% \degree{Doctor of Philosophy \and Master of Science}

% As of the 2007-08 academic year, valid degree months are September, 
% February, or June.  The default is June.
\degreemonth{September}
\degreeyear{2018}
\thesisdate{August 10, 2018}

%% By default, the thesis will be copyrighted to MIT.  If you need to copyright
%% the thesis to yourself, just specify the `vi' documentclass option.  If for
%% some reason you want to exactly specify the copyright notice text, you can
%% use the \copyrightnoticetext command.  
%\copyrightnoticetext{\copyright IBM, 1990.  Do not open till Xmas.}

% If there is more than one supervisor, use the \supervisor command
% once for each.
\supervisor{Leslie K. Norford}{George Macomber (1948) Professor in Construction Management}

% This is the department committee chairman, not the thesis committee
% chairman.  You should replace this with your Department's Committee
% Chairman.
\chairman{Sheila Kennedy}{Professor of Architecture\\Chairman, Department Committee on Graduate Students}

% Make the titlepage based on the above information.  If you need
% something special and can't use the standard form, you can specify
% the exact text of the titlepage yourself.  Put it in a titlepage
% environment and leave blank lines where you want vertical space.
% The spaces will be adjusted to fill the entire page.  The dotted
% lines for the signatures are made with the \signature command.
\maketitle

% The abstractpage environment sets up everything on the page except
% the text itself.  The title and other header material are put at the
% top of the page, and the supervisors are listed at the bottom.  A
% new page is begun both before and after.  Of course, an abstract may
% be more than one page itself.  If you need more control over the
% format of the page, you can use the abstract environment, which puts
% the word "Abstract" at the beginning and single spaces its text.

%% You can either \input (*not* \include) your abstract file, or you can put
%% the text of the abstract directly between the \begin{abstractpage} and
%% \end{abstractpage} commands.

% First copy: start a new page, and save the page number.
\cleardoublepage
% Uncomment the next line if you do NOT want a page number on your
% abstract and acknowledgments pages.
% \pagestyle{empty}
\setcounter{savepage}{\thepage}
\begin{abstractpage}
% $Log: abstract.tex,v $
% Revision 1.1  93/05/14  14:56:25  starflt
% Initial revision
% 
% Revision 1.1  90/05/04  10:41:01  lwvanels
% Initial revision
% 
%
%% The text of your abstract and nothing else (other than comments) goes here.
%% It will be single-spaced and the rest of the text that is supposed to go on
%% the abstract page will be generated by the abstractpage environment.  This
%% file should be \input (not \include 'd) from cover.tex.
Simulation models play an important role in the design, analysis, and optimization of modern energy and environmental systems at building or urban scale. However, due to the extreme complexity of built environments and the sheer number of interacting parameters, it is difficult to obtain an accurate representation of real-world systems. Thus, model calibration and uncertainty analysis hold a particular interest, and it is necessary to evaluate to what degree simulation models are imperfect before implementing them during the decision-making process. In contrast to the extensive literature on the calibration of building performance models, little has been reported on how to automatically calibrate physics-based urban microclimate models. 

This thesis illustrates a general methodology for automatic model calibration and, for the first time, applies it to an urban microclimate system. The study builds upon the previously reported and updated Urban Weather Generator (UWG) to present a deep look into an existing urban district area in downtown Abu Dhabi (UAE) during 2017. Based on 30 candidate inputs covering the meteorological factors, urban characteristics, vegetation variables, and building systems, we performed global sensitivity analysis, Monte Carlo filtering, and optimization-aided calibration on the UWG model. In particular, an online hyper-heuristic evolutionary algorithm (EA) is proposed and developed to accelerate the calibration process. The UWG is a fairly robust simulator to approximate the urban thermal behavior for different seasons. The validation results show that, in single-objective optimization, the online hyper-heuristics can robustly help EA produce quality solutions with smaller uncertainties at much less computational cost. Finally, the resulting calibrated solutions are able to capture weekly-average and hourly diurnal profiles of the urban outdoor air temperature similar to the measurements for certain periods of the year.

\end{abstractpage}

% Additional copy: start a new page, and reset the page number.  This way,
% the second copy of the abstract is not counted as separate pages.
% Uncomment the next 6 lines if you need two copies of the abstract
% page.
% \setcounter{page}{\thesavepage}
% \begin{abstractpage}
% % $Log: abstract.tex,v $
% Revision 1.1  93/05/14  14:56:25  starflt
% Initial revision
% 
% Revision 1.1  90/05/04  10:41:01  lwvanels
% Initial revision
% 
%
%% The text of your abstract and nothing else (other than comments) goes here.
%% It will be single-spaced and the rest of the text that is supposed to go on
%% the abstract page will be generated by the abstractpage environment.  This
%% file should be \input (not \include 'd) from cover.tex.
Simulation models play an important role in the design, analysis, and optimization of modern energy and environmental systems at building or urban scale. However, due to the extreme complexity of built environments and the sheer number of interacting parameters, it is difficult to obtain an accurate representation of real-world systems. Thus, model calibration and uncertainty analysis hold a particular interest, and it is necessary to evaluate to what degree simulation models are imperfect before implementing them during the decision-making process. In contrast to the extensive literature on the calibration of building performance models, little has been reported on how to automatically calibrate physics-based urban microclimate models. 

This thesis illustrates a general methodology for automatic model calibration and, for the first time, applies it to an urban microclimate system. The study builds upon the previously reported and updated Urban Weather Generator (UWG) to present a deep look into an existing urban district area in downtown Abu Dhabi (UAE) during 2017. Based on 30 candidate inputs covering the meteorological factors, urban characteristics, vegetation variables, and building systems, we performed global sensitivity analysis, Monte Carlo filtering, and optimization-aided calibration on the UWG model. In particular, an online hyper-heuristic evolutionary algorithm (EA) is proposed and developed to accelerate the calibration process. The UWG is a fairly robust simulator to approximate the urban thermal behavior for different seasons. The validation results show that, in single-objective optimization, the online hyper-heuristics can robustly help EA produce quality solutions with smaller uncertainties at much less computational cost. Finally, the resulting calibrated solutions are able to capture weekly-average and hourly diurnal profiles of the urban outdoor air temperature similar to the measurements for certain periods of the year.

% \end{abstractpage}

\cleardoublepage

\chapter*{Acknowledgments}

Summary is a skill that may take years to master, a period of time which perhaps I have not fully experienced, thus making my gratitude to those, who have managed to help me relieve my stress and make MIT my beloved home over these two years, with short statements a formidable challenge that goes beyond any of my reasonable articulation. But if there is one thing that I have learned along with the experience that formed this work, it is to challenge the convention, extend the boundary, and believe in yourself.

All my education and research at MIT over these two years have taught me that getting the ultimate correct answer toward any question may never be possible, but it is always possible to answer it to your best knowledge and effort. That answer, I believe, is already useful for many things and people in this world. Here, I shall make an attempt to answer the question again and do what I think to be not shy of impossible -- express my appreciation in a summary.

First and foremost, to my dear advisor, Professor \textit{Leslie K. Norford}, who has absolutely exceeded my expectations, not just in the academic work, but also in his virtue on my journey toward wisdom, pragmatism, and philosophy. In various ways, Les is truly one of the very best role models I could wish to have in my life. While I may be striving to solve problems in the future beyond his wing, I will always look up to him with my highest respect, rectitude, and honor.

There is another person who has surpassed my reasonable notion. I would like to deliver my sincere gratitude to my dear friend, Professor \textit{Afshin Afshari}, who devoted his effort and time to introduce me his splendid research and remotely offer me any help, although we have never met in person. With his exceptional charm, Afshin has always been there for support and advice, which I shall never forget.

In addition, \textit{Yangyang Fu} deserves to be recognized and appreciated. Suffice to say, Yangyang is the one whom at the start of my research during college I thought to be an ordinary labmate but turned out to be a true friend who has supported me any time when I need. He has been a critical part leading to the completion of this thesis, consequently contributing to a lot of my nostalgic memories over time.

I surely would not be honest if I do not admit that there were many others who have provided help for this work. Namely, I would like to thank Professor \textit{Peter R. Armstrong} for supplying the very important data and his insightful comments on my thesis. I would also want to express my great thank-you to Drs. \textit{Peter J. Kempthorne} and \textit{Victor-Emmanual Brunel} for their advice on the uncertainty/sensitivity analysis, to Professors \textit{Dimitris J. Bertsimas} and \textit{Patrick Jaillet} for their advice on the optimization algorithm, and to \textit{Qinzi Luo} for the discussions on the HVAC system.

Finally, I would like to gratefully acknowledge that this research was carried out at the MIT Building Technology Lab with sponsorship supported by the MIT/Masdar Institute Collaborative Program (Ref. No. 02/MI/MIT/CP/11/07633/GEN/G/00), the Leon Hyzen Fellowship, and the George Macomber Chair Scholarship from 2016 to 2018. The views and opinions expressed in this thesis are those of the author and hereby do not necessarily reflect the official policy or position of any agency. Possible errors in the thesis indicate the limitations of the author, who nevertheless hopes the underlying insights could benefit others.


\cleardoublepage

\vspace*{2.5in}
\hspace*{\fill}
\textit{To my dear parents.}
%\hspace*{\fill}
\vspace*{\fill}


%%%%%%%%%%%%%%%%%%%%%%%%%%%%%%%%%%%%%%%%%%%%%%%%%%%%%%%%%%%%%%%%%%%%%%
% -*-latex-*-
