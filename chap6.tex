\chapter{Conclusion}

\AddToShipoutPictureBG*{%
  \AtPageUpperLeft{%
    \hspace*{18.275cm}%
    \raisebox{-3.55cm}{%
      \makebox[0pt][r]{\parbox{\textwidth}{\begin{flushright}\textit{``Now this is not the end. \\It is not even the beginning of the end. \\But it is, perhaps, the end of the beginning.''}\\
      Winston Churchill\end{flushright}}}
}}}%

The extreme complexity of a building or urban system leads to difficulties in estimating the benefits and drawbacks of present and future adaptation strategies to climate change and energy concern. The design, analysis, and optimization of modern building and urban systems may benefit significantly from the implementation of energy and environmental simulation tools at different scales. However, in many cases, studies have revealed large discrepancies between modeled and measured values, which somewhat undermines our confidence in the practical value of these simulation tools. A well-calibrated model is hence one of the key bases for practitioners to perform simulation-based analysis.

This thesis illustrates a general methodology for automatic model calibration and, for the first time, applies it to an existing urban microclimate system. This chapter summarizes the key conclusions.

\section{Summary of contributions}

We started by recognizing that calibration remains an indeterminate and/or over-parametrized problem which could yield non-unique solutions. Hence, it is more reasonable to identify a set of most plausible solutions and to incorporate uncertainty when evaluating and using a calibrated model. In general, we performed global sensitivity analysis, Monte Carlo filtering, and optimization-aided calibration on a microclimate model using the measurements in 2017. Due to the time-constrained nature of engineering applications, an online hyper-heuristic evolutionary algorithm (EA) is proposed and developed to accelerate the calibration process.  Validation of the proposed methods was more of an empirical nature.

The Urban Weather Generator (UWG) \cite{bueno2013urban} is selected as the simulation engine in the present study. The UWG can be used as a physics-based model to produce the microclimate condition at the urban street level by using the meteorological information in available rural weather files. It can also be used as a bottom-up model to estimate the urban energy consumption by aggregating the building stocks. Since the previous version in 2014 \cite{bueno2014computationally}, the UWG has been updated, especially for the urban boundary layer model and the urban canopy-building energy model \cite{yang2016curious}. In general, the newest version aims to make it more physically sound and more capable to handle increasingly detailed building definition.

The District E3 in downtown Abu Dhabi provides a test to the new UWG with an interesting case of heterogeneous building forms located in a tropical or subtropical climate zone. The comparison between the measurements and the predictions by a baseline model from October to December 2016 shows that the UWG can roughly capture the UHI pattern and can produce some plausible values regarding the urban microclimate condition. Thus, together with previous studies \cite{bueno2013urban,bueno2014computationally,nakano2015urban,yang2016curious}, the UWG model can be applied to different climate zones and urban configurations to yield an estimation of the UHI effect.

The regression-based analysis with Monte Carlo sampling is then used to quantify the model uncertainty and to identify significant parameters for summer and winter in 2017, based on 30 candidate inputs from the meteorological factors, urban characteristics, vegetation variables, and building systems. The uncertainty analysis indicates that the UWG is a fairly robust simulator to approximate the thermal behavior of the urban microclimate system in Abu Dhabi for different seasons. It is interesting to observe that both the predicted summer uncertainty band and diurnal variation are generally larger than the winter ones.

The parameter ranking based on the standardized regression coefficients (SRCs) from the linear regression suggests that no vegetation parameter has been identified as a strong parameter in Abu Dhabi during 2017. The most critical parameters are the reference height of the VDM, the UCM-UBL exchange coefficient, the fraction of waste heat into canyon, and the nighttime urban boundary layer height (for winter only). Ironically, these parameters remain the most uncertain among all the input parameters, calling for further investigations into their physical mechanism. Overall, the regression-based analysis is able to identify 12 strong parameters for summer and winter. These 12 parameters are then considered in an optimization-aided calibration process based on the measurements in one summer week and one winter week during 2017.

The proposed online hyper-heuristic EA is roughly twice as fast, on average over 10 trials for the present case study, as the traditional EA to achieve the same objective. In addition, both algorithms are able to identify the sub-ranges of most strong parameters in our current settings, while the solutions from the online hyper-heuristic EA have generally smaller uncertainties. In single-objective optimization, searching solutions mostly in a specific parameter space would act in favor of the online hyper-heuristics, which can thus help EA produce the solutions that are robustly closer to the global optimum with much less computation time.

The solutions obtained from the online hyper-heuristic EA can produce weekly-average diurnal profiles of the urban outdoor air temperature similar to the manually-calibrated baseline solution, which has been extensively investigated for over one year. This is encouraging because a total of only $\sim$1000 simulations (which take at most two days without parallel computing for the present study) could result in an improved urban microclimate model. In addition, the calibrated solutions are able to capture most trends as well as peaks and valleys of the measured data on an hourly basis for certain periods of the year. Despite some yet unexplained behaviors, the calibrated models generally perform well. Therefore, the automatic calibration method proposed in this study is expected to improve the model performance to some extent, both effectively and efficiently.



\section{Discussion and future work}

For simulation-based optimization problems, hyper-heuristics have a catalytic effect in obtaining solutions, but have been overshadowed by the success of parallel computing \cite{zhang2009parallel}. Indeed, some researchers have recently tried to improve the run-time of optimization algorithms via surrogate models \cite{eisenhower2012methodology,boithias2012genetic,vcongradac2012recognition}. However, most studies focused on offline hyper-heuristics where a surrogate or meta-model is trained in advance. This would require additional effort to build a database for a specific case study and the algorithm cannot fully guarantee equivalent quality solutions if the true simulation engine is discarded. Online hyper-heuristics, on the other hand, have a self-updating mechanism without any pre-simulated database to produce well-verified solutions \cite{brownlee2015constrained,xu2016improving}. Therefore, we advocate the use of online hyper-heuristics among relevant research communities that are interested in simulation-based optimization.

The present study only considered a model calibration based on the urban outdoor air temperature via single-objective optimization. Energy use or other urban microclimate conditions (e.g., air humidity) at multi-layer level could be further incorporated into the calibration process if the corresponding data were available at sufficiently high frequency. Given hourly or sub-hourly data of the outdoor microclimate and/or building energy use in an urban neighborhood, one could conduct simultaneous model calibration via multi-objective optimization (MOO) \cite{santos2018evaluating}. The performance of the online hyper-heuristic EA could be further evaluated in this MOO setting. Another direction for future studies would be to develop automatic pattern-based calibration methods \cite{sun2016pattern} or Bayesian calibration methods \cite{heo2012calibration} in order to further improve the results from a purely heuristic-based search. A joint mathematics- and physics-based calibration approach that is able to effectively integrate the actual measurements and computer simulations has the potential to improve the way building and urban systems are designed and operated.

At a more fundamental and philosophical level, any comparison between the predicted and measured performance could be viewed in terms of not only how well the simulation agrees with the measurement, but also \textit{whether the simulation program is good enough for its intended purposes} \cite{reddy2007calibrating}. The question of interest is, then, how to determine a ``good enough'' solution? In particular, there should be a broad consensus among the scientific and engineering communities when it comes to the specification of the accuracy bound at different scales for the calibration to be deemed satisfactory. Scientists prefer to describe such bound as \textit{uncertainty}, which is one key idea we want to deliver in this thesis. The input uncertainties indicate the difficulties in capturing the inherent physical properties (e.g., model parameter values) during a specific simulation setting. It is assumed that, if the model is calibrated within the prescribed criteria, it seems closer to the physical reality as the input parameters (i.e., the ``knobs'') are tuned properly.

However, just because all the selected knobs yield a desired output, we cannot guarantee that each knob is tuned correctly \cite{reddy2007calibrating}. Most simulation models have many degrees of freedom and, with judicious fiddling, can be easily manipulated to produce any desired behavior with both plausible model structures and parameter values. More often than not, this calibration process can be seen as GIGO (garbage in, garbage out) \cite{saltelli2008global}. The reason may lie in the fact that, during an optimization process, we usually obtain one specific parameter combination leading to a ``local'' optimum within the search space. Thus, the output (or solution) uncertainties, which demonstrate the consistency of a calibration process, should also be considered with great care after calibration. Earlier, Kaplan et al. \cite{kaplan1990reconciliation} looked at this issue and concluded that one can never hope to identify the parameters correctly, in part because we do not know what is correct. Given this unanswerable (and maybe hopeless) situation, limiting ourselves to one plausible solution would be quite misleading. It is hence much more reasonable to incorporate both the input and output uncertainties when evaluating and using a calibrated model. 

The present study has not considered the parameters of the rural measurement, which might be an interest of research in the future if sufficient data is available. In addition, to incorporate the actual effect on an urban system due to occupancy, we might need to combine the UWG with reliable stochastic occupant behavior models \cite{hong2017ten}. Finally, the energy flow process modeled by the current UWG from the rural station to the urban canopy may not be a complete mechanism to depict the physics of the urban microclimate, even though most UWG outputs seem reasonable and the intended purposes of UWG have been fulfilled to some extent. Numerical CFD simulations might be further needed to illustrate whether the actual interaction between the air masses in the rural and urban area has been accurately formulated for a specific city.

%\let\cleardoublepage\clearpage
\newpage
\mbox{}