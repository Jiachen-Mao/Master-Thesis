% $Log: abstract.tex,v $
% Revision 1.1  93/05/14  14:56:25  starflt
% Initial revision
% 
% Revision 1.1  90/05/04  10:41:01  lwvanels
% Initial revision
% 
%
%% The text of your abstract and nothing else (other than comments) goes here.
%% It will be single-spaced and the rest of the text that is supposed to go on
%% the abstract page will be generated by the abstractpage environment.  This
%% file should be \input (not \include 'd) from cover.tex.
Simulation models play an important role in the design, analysis, and optimization of modern energy and environmental systems at building or urban scale. However, due to the extreme complexity of built environments and the sheer number of interacting parameters, it is difficult to obtain an accurate representation of real-world systems. Thus, model calibration and uncertainty analysis hold a particular interest, and it is necessary to evaluate to what degree simulation models are imperfect before implementing them during the decision-making process. In contrast to the extensive literature on the calibration of building performance models, little has been reported on how to automatically calibrate physics-based urban microclimate models. 

This thesis illustrates a general methodology for automatic model calibration and, for the first time, applies it to an urban microclimate system. The study builds upon the previously reported and updated Urban Weather Generator (UWG) to present a deep look into an existing urban district area in downtown Abu Dhabi (UAE) during 2017. Based on 30 candidate inputs covering the meteorological factors, urban characteristics, vegetation variables, and building systems, we performed global sensitivity analysis, Monte Carlo filtering, and optimization-aided calibration on the UWG model. In particular, an online hyper-heuristic evolutionary algorithm (EA) is proposed and developed to accelerate the calibration process. The UWG is a fairly robust simulator to approximate the urban thermal behavior for different seasons. The validation results show that, in single-objective optimization, the online hyper-heuristics can robustly help EA produce quality solutions with smaller uncertainties at much less computational cost. Finally, the resulting calibrated solutions are able to capture weekly-average and hourly diurnal profiles of the urban outdoor air temperature similar to the measurements for certain periods of the year.
